\documentclass[margin,line]{res}
\usepackage{hyperref}
\usepackage{url}
\oddsidemargin -.5in
\evensidemargin -.5in
\textwidth=6.0in
\itemsep=0in
\parsep=0in
\topmargin=0in
\topskip=0in
 
\newenvironment{list1}{
  \begin{list}{\ding{113}}{%
      \setlength{\itemsep}{0in}
      \setlength{\parsep}{0in} \setlength{\parskip}{0in}
      \setlength{\topsep}{0in} \setlength{\partopsep}{0in}
      \setlength{\leftmargin}{0.17in}}}{\end{list}}
\newenvironment{list2}{
  \begin{list}{$\bullet$}{%
      \setlength{\itemsep}{0in}
      \setlength{\parsep}{0in} \setlength{\parskip}{0in}
      \setlength{\topsep}{0in} \setlength{\partopsep}{0in}
      \setlength{\leftmargin}{0.2in}}}{\end{list}}

\usepackage{hyperref}
\hypersetup{
    colorlinks=true,
    linkcolor=black,
    filecolor=magenta,      
    urlcolor=cyan,
}

\usepackage{etaremune}
\usepackage{mathabx}

    
\begin{document}

\name{\LARGE \hspace{7.5cm}Sean K. Terry}

\begin{resume}
\section{\sc Personal}

\vspace{.05in}
\begin{tabular}{@{}p{3.5in}p{3in}}
Department of Astronomy      & Email: \,\,\,sean.terry@berkeley.edu \\
501 Campbell Hall \#3411       &  Github: \hyperlink{https://github.com/skterry}{skterry} \\
Berkeley, CA 94720       & \hyperlink{http://w.astro.berkeley.edu/~sean.terry}{http://w.astro.berkeley.edu/$\sim$sean.terry}\\
 \\
\end{tabular}

\section{\sc Appointments}
%%%%%%
{\bf Postdoctoral Scholar}, University of California, Berkeley  \hfill      {November 2020 $-$ Present}\\
%%%%%%


\section{\sc Education}
{\bf The Catholic University of America}, Ph.D., Physics  \hfill 2020\\
{\bf The Catholic University of America}, M.S., Physics  \hfill 2018\\
{\bf George Mason University}, B.S., Astronomy/Physics \hfill 2015\\
{\bf Northern Virginia Community College}, A.S., Gen. Science \hfill 2012\\

\section{\sc Research\\ Areas}
Gravitational microlensing by exoplanets  \\
Astrometric microlensing \\
Galactic bulge stellar kinematics and populations  \\
Computational astrophysics \\
Adaptive optics \\


\section{\sc Service \&\\ Professional Activities}
\textbf{Professional Activities}\\
Project Science Team $-$ Keck All-Sky Precision Adaptive Optics (KAPA) \hfill 2020$-$present\\
Representative $-$ Annual GSFC Administrator's Congressional Visits \hfill 2016\\
Local Organizing Committee $-$ 19th International Conference on Microlensing \hfill 2015\\
\\
\textbf{Professional Memberships}\\
Member $-$ American Astronomical Society (AAS) \hfill 2015$-$present\\
Member $-$ Society for Personality and Social Psychology (SPSP) \hfill 2017$-$2020\\
Member $-$ Seers Exoplanet Environments Collaboration (SEEC) \hfill 2016$-$2020\\
\\
\textbf{Panels and Reviews}\\
\textit{TESS} Cycle 4 \hfill 2021\\

\section{\sc Teaching Experience}
Guest Lecturer (American U.), \textit{Complex Problems Seminar: Exoplanets in Fact \& Fiction} \hfill 2019 \\
Teaching Assistant (GMU), \textit{Astronomy for non-STEM Majors} \hfill 2014 \\
Teaching Assistant (GMU), \textit{Introduction to Astrophysics} \hfill 2013 \\

\section{\sc Mentoring}
\textbf{NASA Goddard Summer Interns} \\
Ishaan Gandhi $-$ current: Harvey Mudd College \hfill 2016 \\
Anshula Gandhi $-$ MIT (graduated) \hfill 2015\\
Mackenzie Kynoch $-$ Dartmouth (graduated) \hfill 2015 \\

\section{\sc Outreach}
Guest Speaker, STEM-Day, Garfield High School, Woodbridge, VA \hfill 2017 \\
CUA Booth, Annual Astronomy Festival on the Mall, Washington, DC \hfill 2015$-$2017 \\
Proctor, GMU Public Observing Nights, Fairfax, VA \hfill 2013$-$2015 \\
MATHCOUNTS ambadassor \& judge, TJ High School, VA \hfill 2013$-$2014 \\

\section{\sc Grants \\ Awarded}
Hubble Space Telescope Cycle 28 Grant \#16509 \\
``\textit{Detection of the Astrometric Microlensing Signal by the Binary Black Hole Candidate MOA-2019-BLG-284}" \\
Principle Investigator: S. K. Terry \\
March 9, 2021 $-$ November 31, 2021

Keck Semester 2021A \\
``\textit{Testing Core Accretion with Microlens Planet Host Star Masses}" \\
Principle Investigator: D. P. Bennett\\
May 17, 2021 $-$ July 13, 2021

Keck Semester 2020B \\
``\textit{Confirmation of a Massive Black Hole Microlens Candidate}" \\
Principle Investigator: D. P. Bennett\\
August 2, 2020 $-$ August 11, 2020

\section{\sc Observing}
HST (WFC3/UVIS), 4 orbits \hfill 2021\\
Keck 10m (NIRC2/OSIRIS), 8 nights \hfill 2019$-$2021\\
GMU 0.8m, 24 nights \hfill 2013$-$2015 \\
%\textit{Testing Core Accretion with Microlens Planet Host Star Masses}, Keck 10m, 2 nights \hfill 2021A \\
%\textit{Confirmation of a Massive Black Hole Candidate}, Keck 10m, assisted 3 nights \hfill 2020B \\
%\textit{Astrometric Microlensing of Black Hole Candidates}, Keck 10m, assisted 1 night \hfill 2020A \\
%\textit{Development of the WFIRST Mass Measurement Method}, Keck 10m, assisted 2 nights \hfill 2019A \\

\section{\sc Talks \&\\ Proceedings}
\textbf{12 talks (3 invited$^{\dagger}$, 9 public)}\\
\begin{etaremune}
\item $^{\dagger}$``A Sub-Saturn Exoplanet Inside the Mass Desert Predicted by Core Accretion", University of Maryland, November 2020
\item ``Roman Space Telescope Mass-measurement Method Determines a Mass of $66 \pm 8M_{\Earth}$ for MOA-2009-BLG-319Lb", Chesapeake Bay Area Exoplanet Meeting (chExo) \#8, June 2020

\item $^{\dagger}$``Comparing HST Observations of Bulge Stars to Galactic Population Synthesis Models in Preparation for the WFIRST Microlensing Survey", NASA GSFC, November 2019 
\item $^{\dagger}$``Probing the Galactic Bulge Stellar Population as Precursor Science for WFIRST", University of Maryland, May 2018

\item ``Preparing for the WFIRST Microlensing Survey: Stellar Populations in the Galactic Bulge", George Mason University, November 2017

\item ``Precursor Science for the WFIRST Mission", Sagan Exoplanet Summer Workshop, Caltech, August 2017

\item ``A Deep Study of the Stanek Field as Precursor Science for the WFIRST Microlensing Field of Regard", George Washington University, July 2017

\item ``Bayesian Modeling of Gravitational Microlensing Events", George Washington University, June 2016

\item ``A New Toolkit for Modeling Gravitational Microlensing Events", The College of William \& Mary, March 2016

\item ``Exoplanet Detection with WFIRST", The Catholic University of America, July 2015

\item ``A New Near-IR Luminosity Function in the WFIRST Microlensing Fields", 19th International Conference on Gravitational Microlensing, January 2015

\item ``Light Curve Analysis of HD 189733b, WASP-33b and KELT-1b", George Mason University, November 2013 \\

\end{etaremune}

\section{\sc Publications}
\textbf{10 total (5 first author)}\\
$^{\dagger}$ = unrefereed publications \\
\begin{etaremune}
\item \textbf{Terry, S. K.}, Bhattacharya, A., Bennett, D. P., Bond, I.A., et al. ``Using Keck Adaptive Optics to Break the Degeneracies for OGLE-2011-BLG-0950", 2021, \textit{in prep}

\item Bhattacharya, A., Bennett, D. P., Beaulieu, J., \& 11 coauthors including \textbf{Terry, S. K.}, ``MOA-2007-BLG-400Lb: A Super-Jupiter Mass Planet Orbiting a Galactic Bulge K-dwarf Revealed by Keck Adaptive Optics Imaging", 2021, \textit{submitted to AJ}

\item Blackman, J., Beaulieu, J., Bennett, D. P., \& 11 coauthors including {\bf Terry, S. K.}, ``A Jovian Analog Orbiting a White Dwarf Star", 2021, \textit{submitted to Nature}

\item $^{\dagger}${\bf Terry, S. K.}, ``Breaking a New Degeneracy in High Magnification Microlensing Events", 2021, \textit{American Astronomical Society}, 237, 218.03

\item \textbf{Terry, S. K.}, Bhattacharya, A., Bennett, D. P., Bond, I.A., et al. ``MOA-2009-BLG-319Lb: A Sub-Saturn Planet Inside the Predicted Mass Desert", 2021, \textit{AJ}, 161, 54

\item {\bf Terry, S. K.}, Barry, R. K., Bennett, D. P., Bhattacharya, A., Anderson, J., Penny, M. T., ``Comparing Observed Stellar Kinematics and Surface Densities in a Low Latitude Bulge Field to Galactic Population Synthesis Models", 2020, \textit{ApJ}, 889, 126

\item Bennett, D. P., Bhattacharya, A., Beaulieu, J., \& 9 coauthors including {\bf Terry, S. K.}, ``Keck Observations Confirm a Super-Jupiter Planet Orbiting M-dwarf OGLE-2005-BLG-071L", 2020, \textit{AJ}, 159, 68

\item $^{\dagger}${\bf Terry, S. K.}, ``Direct Mass Measurements for Planets Discovered by Gravitational Microlensing", 2020, \textit{American Astronomical Society}, 235, 402.01

\item Bennett, D. P., Bhattacharya, A., Anderson, J., \& 15 coauthors including {\bf Terry, S. K.}, ``Confirmation of the Planetary Microlensing Signal and Star and Planet Mass Determinations for Event OGLE-2005-BLG-169", 2015, \textit{ApJ}, 808, 169

\item $^{\dagger}$Gilbert, E., {\bf Terry, S. K.}, Pfeifle, R, ``A New Luminosity Function for Stars in the Galactic Bulge", 2015, \textit{American Astronomical Society}, 225, 102.02 \\

\end{etaremune}

\section{\sc Skills}
Python, Fortran, IDL, gnu, Git, Bash

%%%%%%%%%%%%%%%%%%%%%%%%%
%%%%%%%%%%%
%%%%%%%%%%%%%%%%

\section{\sc References }
Available upon request.

\end{resume}
\end{document}




